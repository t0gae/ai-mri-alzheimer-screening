\documentclass[11pt]{article}

\usepackage[a4paper,margin=1in]{geometry}
\usepackage{graphicx}
\usepackage{float}
\usepackage{subcaption}
\usepackage{amsmath,amsfonts}
\usepackage{booktabs}
\usepackage{hyperref}

\title{
Screening-Oriented 3D Convolutional Neural Network \\
for Alzheimer's Disease Detection from Structural MRI
}

\author{
Georgii A. Erokhin \\
\small Department of Information Security \\
\small College of Information Technologies \\
\small \texttt{georgii.erokhin@gmail.com}
}

\date{\today}

\begin{document}
\maketitle

% =========================
\begin{abstract}
Early detection of Alzheimer's disease (AD) is critical for timely clinical intervention and disease management.
Structural magnetic resonance imaging (sMRI) provides non-invasive biomarkers of neurodegeneration, but manual analysis is time-consuming and subject to inter-observer variability.

In this work, we present a lightweight 3D convolutional neural network (CNN) for binary classification of Alzheimer's disease versus cognitively normal controls using volumetric MRI scans.
The model is explicitly optimized for screening applications by prioritizing sensitivity through aggressive class-weighted training.
Our architecture consists of three convolutional blocks with batch normalization, progressive dropout regularization, and global average pooling.

Experiments on a held-out test set of 162 subjects demonstrate an AUC-ROC of 0.824 and a recall of 0.943 for dementia detection, with only 3 missed cases out of 53.
While overall accuracy is moderate (67.9\%), the high sensitivity makes the proposed approach suitable as a first-pass automated screening tool to assist clinical decision-making.

\end{abstract}

% =========================
\section{Introduction}

Alzheimer's disease (AD) is the most common cause of dementia, affecting more than 55 million people worldwide, with prevalence expected to increase significantly due to population aging.
Early diagnosis is essential for treatment planning, clinical trials, and patient care.
Structural MRI (sMRI) reveals characteristic patterns of brain atrophy in AD, particularly in the hippocampus, entorhinal cortex, and temporal lobes~\cite{jack2010hypothetical}.

Traditional neuroimaging pipelines rely on handcrafted features and region-of-interest volumetry, which require expert knowledge and substantial preprocessing.
Recent advances in deep learning have enabled end-to-end learning directly from medical images, achieving promising results in brain disease classification~\cite{lecun2015deep, litjens2017survey}.

However, many existing deep learning approaches prioritize overall accuracy, which may be suboptimal for clinical screening where false negatives are significantly more costly than false positives.
In this study, we explicitly design and evaluate a screening-oriented 3D CNN that emphasizes high sensitivity for dementia detection.

Our main contributions are:
\begin{itemize}
    \item A lightweight 3D CNN architecture suitable for limited neuroimaging datasets
    \item Class-weighted training strategy to maximize dementia recall
    \item Comprehensive evaluation with clinically motivated interpretation
\end{itemize}

% =========================
\section{Methodology}

\subsection{Dataset and Preprocessing}

We utilize a dataset of 520 structural MRI brain scans with binary labels: cognitively normal (CN) and Alzheimer's disease (AD).
The dataset is split using stratified sampling into training, validation, and test sets:

\begin{itemize}
    \item Training set: 333 subjects (64\%)
    \item Validation set: 84 subjects (16\%)
    \item Test set: 162 subjects (31\%, held-out)
\end{itemize}

The test set contains 109 normal controls (67\%) and 53 dementia cases (33\%), reflecting the class imbalance in the full dataset.

All MRI volumes are preprocessed using standard neuroimaging procedures:
\begin{enumerate}
    \item Skull stripping to remove non-brain tissue
    \item Bias field correction for intensity inhomogeneity
    \item Affine registration to MNI152 template space
    \item Resampling to $64 \times 64 \times 64$ isotropic resolution
    \item Intensity normalization to the $[0,1]$ range
\end{enumerate}

\subsection{Problem Formulation}

Let $\mathcal{D} = \{(X_i, y_i)\}_{i=1}^N$ denote a dataset of structural MRI scans,
where $X_i \in \mathbb{R}^{64 \times 64 \times 64}$ represents a 3D brain volume
and $y_i \in \{0,1\}$ is the corresponding class label,
with $0$ indicating cognitively normal control and $1$ indicating dementia.

The task is to learn a function
\[
f_\theta: \mathbb{R}^{64 \times 64 \times 64} \rightarrow [0,1]
\]
parameterized by $\theta$, which estimates the posterior probability
$P(y = 1 \mid X)$.

\subsection{Model Architecture}

The proposed 3D CNN consists of three convolutional blocks followed by a fully connected classifier.
Each convolutional block includes 3D convolution, batch normalization, ReLU activation, max pooling, and dropout.

\textbf{Block 1:}
\begin{itemize}
    \item Conv3D (32 filters, $3 \times 3 \times 3$, same padding)
    \item Batch Normalization
    \item ReLU activation
    \item MaxPooling3D ($2 \times 2 \times 2$)
    \item Dropout (0.1)
\end{itemize}

\textbf{Block 2:}
\begin{itemize}
    \item Conv3D (64 filters, $3 \times 3 \times 3$, same padding)
    \item Batch Normalization
    \item ReLU activation
    \item MaxPooling3D ($2 \times 2 \times 2$)
    \item Dropout (0.2)
\end{itemize}

\textbf{Block 3:}
\begin{itemize}
    \item Conv3D (128 filters, $3 \times 3 \times 3$, same padding)
    \item Batch Normalization
    \item ReLU activation
    \item Global Average Pooling 3D
    \item Dropout (0.3)
\end{itemize}

\textbf{Classifier:}
\begin{itemize}
    \item Dense (128 units, ReLU)
    \item Dropout (0.3)
    \item Dense (1 unit, sigmoid)
\end{itemize}

The progressive dropout strategy provides increasing regularization in deeper layers.
Global average pooling reduces the number of parameters and improves generalization compared to flattening.
The total number of trainable parameters is approximately 450,000.

\subsection{Loss Function and Class Imbalance Handling}

Due to class imbalance in the dataset (approximately 2:1 ratio of normal to dementia),
we employ weighted binary cross-entropy loss.
The loss for a single sample is defined as:
\[
\mathcal{L}(y, \hat{y}) =
- w_1 \, y \log(\hat{y})
- w_0 \, (1 - y) \log(1 - \hat{y}),
\]
where $\hat{y} = f_\theta(X)$ is the predicted probability of dementia,
and $w_0$, $w_1$ are class weights.

In our experiments, we set:
\[
w_0 = 1.0, \quad w_1 = 5.0,
\]
placing higher penalty on false negatives (missed dementia cases).
This design choice reflects clinical priorities,
where missing a dementia diagnosis has more serious consequences than a false alarm.

\subsection{Training Configuration}

The model is optimized using the Adam optimizer~\cite{lecun2015deep} with an initial learning rate of $5 \times 10^{-4}$.
Training is performed for up to 100 epochs with the following callbacks:

\begin{itemize}
    \item \textbf{Early stopping:} Monitors validation AUC with patience of 15 epochs
    \item \textbf{ReduceLROnPlateau:} Reduces learning rate by factor 0.5 when validation AUC plateaus (patience 5 epochs)
    \item \textbf{Model checkpoint:} Saves best model weights based on validation AUC
\end{itemize}

A batch size of 4 was used due to GPU memory constraints when processing 3D volumes.
The model converged after 35 epochs, with early stopping restoring weights from epoch 20.

\subsection{Evaluation Metrics}

Performance is evaluated using standard classification metrics:

\begin{itemize}
    \item \textbf{Accuracy:} Overall classification correctness
    \item \textbf{Precision:} Positive predictive value for dementia class
    \item \textbf{Recall (Sensitivity):} True positive rate, $\text{TPR} = \frac{TP}{TP + FN}$
    \item \textbf{F1-score:} Harmonic mean of precision and recall
    \item \textbf{AUC-ROC:} Area under the receiver operating characteristic curve
\end{itemize}

The ROC curve plots TPR versus FPR:
\begin{equation}
\text{FPR} = \frac{FP}{FP + TN}
\end{equation}

AUC provides a threshold-independent measure of discriminative performance.

% =========================
\section{Results}

\subsection{Overall Performance}

Table~\ref{tab:metrics} summarizes the model's performance on the held-out test set of 162 subjects.

\begin{table}[H]
\centering
\caption{Performance metrics on test set (n=162).}
\label{tab:metrics}
\begin{tabular}{@{}lc@{}}
\toprule
\textbf{Metric} & \textbf{Value} \\ \midrule
Accuracy        & 0.6790         \\
Precision       & 0.5051         \\
Recall          & 0.9434         \\
F1-Score        & 0.6579         \\
AUC-ROC         & 0.8238         \\ \bottomrule
\end{tabular}
\end{table}

The model achieves high recall (0.943), correctly identifying 50 out of 53 dementia cases.
Only 3 dementia cases are misclassified as normal (false negative rate of 5.7\%).
The AUC-ROC of 0.824 indicates strong discriminative ability.

\subsection{ROC Curve Analysis}

Figure~\ref{fig:roc} presents the receiver operating characteristic curve.
The curve shows the trade-off between sensitivity and specificity across all classification thresholds.
The area under the curve (AUC = 0.824) substantially exceeds random performance (AUC = 0.5)
and approaches excellent classification (AUC $>$ 0.9).

\begin{figure}[H]
\centering
\includegraphics[width=0.65\linewidth]{figures/roc_curve_plotly.png}
\caption{ROC curve on the held-out test set. The model achieves AUC = 0.824, significantly outperforming random classification (dashed diagonal).}
\label{fig:roc}
\end{figure}

\subsection{Confusion Matrix}

Figure~\ref{fig:cm} presents the confusion matrix.
Out of 162 test subjects, the model correctly classified 110 cases (67.9\% accuracy).

\begin{figure}[H]
\centering
\includegraphics[width=0.55\linewidth]{figures/confusion_matrix_plotly.png}
\caption{Confusion matrix on test set. True labels on rows, predicted labels on columns.}
\label{fig:cm}
\end{figure}

The detailed breakdown is:
\begin{itemize}
    \item \textbf{True Negatives (TN):} 60 (55.0\% of normal controls correctly identified)
    \item \textbf{False Positives (FP):} 49 (45.0\% of normal controls misclassified as dementia)
    \item \textbf{False Negatives (FN):} 3 (5.7\% of dementia cases missed)
    \item \textbf{True Positives (TP):} 50 (94.3\% of dementia cases correctly identified)
\end{itemize}

\subsection{Per-Class Performance}

Table~\ref{tab:perclass} shows detailed metrics for each class.

\begin{table}[H]
\centering
\caption{Per-class performance metrics.}
\label{tab:perclass}
\begin{tabular}{@{}lcccc@{}}
\toprule
\textbf{Class} & \textbf{Precision} & \textbf{Recall} & \textbf{F1-Score} & \textbf{Support} \\ \midrule
Normal         & 0.9524             & 0.5505          & 0.6977            & 109              \\
Dementia       & 0.5051             & 0.9434          & 0.6579            & 53               \\ \midrule
Macro Avg      & 0.7287             & 0.7469          & 0.6778            & 162              \\
Weighted Avg   & 0.8060             & 0.6790          & 0.6847            & 162              \\ \bottomrule
\end{tabular}
\end{table}

The normal class achieves high precision (0.95), meaning that when the model predicts ``normal'', it is correct 95\% of the time.
Conversely, the dementia class achieves high recall (0.94), meaning the model successfully detects 94\% of all dementia cases.

% =========================
\section{Discussion}

\subsection{Clinical Utility}

The results demonstrate that aggressive class weighting (5:1 ratio) effectively prioritizes dementia detection, achieving high sensitivity at the cost of reduced specificity.
This trade-off is appropriate for screening applications, where missing a dementia case has more serious consequences than generating a false alarm.

In a clinical workflow, the model could serve as a first-pass screening tool:
\begin{enumerate}
    \item All 162 scans are processed through the CNN
    \item The model flags 99 cases (50 TP + 49 FP) as potentially having dementia
    \item Clinicians review only the 99 flagged cases (39\% reduction in workload)
    \item Only 3 out of 53 true dementia cases are missed (5.7\% false negative rate)
\end{enumerate}

While 49 false positives may seem high, this is acceptable in screening contexts where:
\begin{itemize}
    \item Subsequent clinical evaluation by experts is standard practice
    \item The cost of missing a dementia diagnosis far exceeds the cost of unnecessary follow-up
    \item The alternative is manual review of all 162 scans
\end{itemize}

\subsection{Comparison with Prior Work}

Our AUC of 0.824 is competitive with recent deep learning approaches for AD classification:
\begin{itemize}
    \item Liu et al.~\cite{liu2018landmark}: AUC 0.84 using landmark-based multi-instance learning
    \item Oh et al.~\cite{oh2019classification}: Accuracy 0.88 using transfer learning on larger datasets
    \item Korolev et al.~\cite{korolev20173d}: Accuracy 0.80 with residual 3D CNNs
\end{itemize}

While our overall accuracy (67.9\%) is moderate, the recall of 0.943 exceeds typical values reported in prior work.
This reflects our explicit design choice to prioritize sensitivity for screening applications.

\subsection{Design Trade-offs}

The 5:1 class weighting creates a deliberate trade-off:
\begin{itemize}
    \item \textbf{High dementia recall (0.94):} Catches nearly all dementia cases
    \item \textbf{Low dementia precision (0.51):} Half of dementia predictions are false positives
    \item \textbf{High normal precision (0.95):} When predicting normal, usually correct
    \item \textbf{Low normal recall (0.55):} Many normals incorrectly flagged as dementia
\end{itemize}

Alternative configurations (e.g., equal class weights or different decision thresholds) would shift this trade-off, potentially increasing accuracy but reducing sensitivity.

\subsection{Limitations and Future Work}

Several limitations should be acknowledged:

\begin{itemize}
    \item \textbf{Dataset size:} 520 subjects is modest for deep learning; larger datasets may improve generalization
    \item \textbf{Binary classification:} Does not capture disease severity or mild cognitive impairment (MCI)
    \item \textbf{No external validation:} Performance on independent datasets (ADNI, OASIS) is unknown
    \item \textbf{Lack of interpretability:} Model does not explain which brain regions drive predictions
\end{itemize}

Future work will explore:
\begin{enumerate}
    \item Attention mechanisms to identify disease-relevant brain regions
    \item Multi-class classification including MCI and disease stages
    \item External validation on public datasets
    \item Grad-CAM visualizations for model interpretability
    \item Integration with clinical data (cognitive scores, biomarkers)
\end{enumerate}

% =========================
\section{Conclusion}

We presented a screening-oriented 3D CNN for Alzheimer's disease detection from structural MRI.
By prioritizing sensitivity through class-weighted training, the model achieves high recall (0.943) and competitive discriminative performance (AUC = 0.824) on a held-out test set.

Key findings include:
\begin{itemize}
    \item Only 3 out of 53 dementia cases missed (5.7\% false negative rate)
    \item Strong AUC (0.824) indicates good class separation
    \item Lightweight architecture (450K parameters) suitable for limited datasets
    \item Trade-off between sensitivity and specificity appropriate for screening
\end{itemize}

These results demonstrate the potential utility of the approach as an automated first-pass screening tool to support clinical workflows.
The high sensitivity addresses a critical clinical need, enabling early detection while reducing manual review burden.

\bibliographystyle{ieeetr}
\bibliography{references}

\end{document}
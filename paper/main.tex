\documentclass[11pt]{article}

\usepackage[a4paper,margin=1in]{geometry}
\usepackage{graphicx}
\usepackage{float}
\usepackage{subcaption}
\usepackage{amsmath,amsfonts}
\usepackage{booktabs}
\usepackage{hyperref}

\title{
Screening-Oriented 3D Convolutional Neural Network \\
for Alzheimer's Disease Detection from Structural MRI
}

\author{
Georgii A. Erokhin \\
\small Department of Information Security \\
\small College of Information Technologies \\
\small \texttt{georgii.erokhin@gmail.com}
}

\date{\today}

\begin{document}
\maketitle

% =========================
\begin{abstract}
Early detection of Alzheimer's disease (AD) is critical for timely clinical intervention and disease management.
Structural magnetic resonance imaging (sMRI) provides non-invasive biomarkers of neurodegeneration, but manual analysis is time-consuming and subject to inter-observer variability.

In this work, we present a lightweight 3D convolutional neural network (CNN) for binary classification of Alzheimer's disease versus cognitively normal controls using volumetric MRI scans.
The model is explicitly optimized for screening applications by prioritizing sensitivity through aggressive class-weighted training.
Our architecture consists of three convolutional blocks with batch normalization, progressive dropout regularization, and global average pooling.

Experiments on a held-out test set of 162 subjects demonstrate an AUC-ROC of 0.824 and a recall of 0.943 for dementia detection, with only 3 missed cases out of 53.
While overall accuracy is moderate (0.679), the high sensitivity makes the proposed approach suitable as a first-pass automated screening tool to assist clinical decision-making.

\end{abstract}

% =========================
\section{Introduction}

Alzheimer's disease (AD) is the most common cause of dementia, affecting more than 55 million people worldwide, with prevalence expected to increase significantly due to population aging.
Early diagnosis is essential for treatment planning, clinical trials, and patient care.
Structural MRI (sMRI) reveals characteristic patterns of brain atrophy in AD, particularly in the hippocampus, entorhinal cortex, and temporal lobes~\cite{jack2010hypothetical}.

Traditional neuroimaging pipelines rely on handcrafted features and region-of-interest volumetry, which require expert knowledge and substantial preprocessing.
Recent advances in deep learning have enabled end-to-end learning directly from medical images, achieving promising results in brain disease classification~\cite{lecun2015deep, litjens2017survey}.

However, many existing deep learning approaches prioritize overall accuracy, which may be suboptimal for clinical screening where false negatives are significantly more costly than false positives.
In this study, we explicitly design and evaluate a screening-oriented 3D CNN that emphasizes high sensitivity for dementia detection.

Our main contributions are:
\begin{itemize}
    \item A lightweight 3D CNN architecture suitable for limited neuroimaging datasets
    \item Class-weighted training strategy to maximize dementia recall
    \item Comprehensive evaluation with clinically motivated interpretation
\end{itemize}

% =========================
\section{Methodology}

\subsection{Dataset and Preprocessing}

We utilize a dataset of 520 structural MRI brain scans with binary labels: cognitively normal (CN) and Alzheimer's disease (AD).
The dataset is split using stratified sampling into training, validation, and test sets:

\begin{itemize}
    \item Training set: 333 subjects (64\%)
    \item Validation set: 84 subjects (16\%)
    \item Test set: 162 subjects (31\%)
\end{itemize}

The held-out test set contains 109 normal controls and 53 dementia cases.

All MRI volumes are preprocessed using standard neuroimaging procedures:
\begin{enumerate}
    \item Skull stripping to remove non-brain tissue
    \item Bias field correction for intensity inhomogeneity
    \item Affine registration to MNI152 template space
    \item Resampling to $64 \times 64 \times 64$ isotropic resolution
    \item Intensity normalization to the $[0,1]$ range
\end{enumerate}

\subsection{Problem Formulation}

Let $\mathcal{D} = \{(X_i, y_i)\}_{i=1}^N$ denote a dataset of structural MRI scans,
where $X_i \in \mathbb{R}^{64 \times 64 \times 64}$ represents a 3D brain volume
and $y_i \in \{0,1\}$ is the corresponding class label,
with $0$ indicating cognitively normal control and $1$ indicating dementia.

The task is to learn a function
\[
f_\theta: \mathbb{R}^{64 \times 64 \times 64} \rightarrow [0,1]
\]
parameterized by $\theta$, which estimates the posterior probability
$P(y = 1 \mid X)$.

\subsection{Model Architecture}

The proposed 3D CNN consists of three convolutional blocks followed by a fully connected classifier.
Each convolutional block includes 3D convolution, batch normalization, ReLU activation, max pooling, and dropout.

\textbf{Block 1:}
\begin{itemize}
    \item Conv3D (32 filters, $3 \times 3 \times 3$)
    \item Batch Normalization
    \item ReLU activation
    \item MaxPooling3D ($2 \times 2 \times 2$)
    \item Dropout (0.1)
\end{itemize}

\textbf{Block 2:}
\begin{itemize}
    \item Conv3D (64 filters)
    \item Batch Normalization
    \item ReLU activation
    \item MaxPooling3D
    \item Dropout (0.2)
\end{itemize}

\textbf{Block 3:}
\begin{itemize}
    \item Conv3D (128 filters)
    \item Batch Normalization
    \item ReLU activation
    \item Global Average Pooling
    \item Dropout (0.3)
\end{itemize}

The classifier head consists of a dense layer with 128 units followed by a sigmoid output.
The total number of trainable parameters is approximately 450,000.

\subsection{Loss Function and Class Imbalance Handling}

Due to class imbalance in the dataset, a weighted binary cross-entropy loss was employed.
The loss for a single sample is defined as:
\[
\mathcal{L}(y, \hat{y}) =
- w_1 \, y \log(\hat{y})
- w_0 \, (1 - y) \log(1 - \hat{y}),
\]
where $\hat{y} = f_\theta(X)$ is the predicted probability of dementia,
and $w_0$, $w_1$ are class weights.

In our experiments, we set:
\[
w_0 = 1.0, \quad w_1 = 5.0,
\]
placing higher penalty on false negatives.
This design choice reflects clinical priorities,
where missing a dementia diagnosis is more critical than generating false alarms.

\subsection{Training Configuration}

The model is optimized using the Adam optimizer with an initial learning rate of $5 \times 10^{-4}$.
Training is performed for up to 100 epochs with early stopping based on validation AUC, using a patience of 15 epochs.
A batch size of 4 was used due to GPU memory constraints when processing 3D volumes.

\subsection{Evaluation Metrics}

Performance is evaluated using accuracy, precision, recall (sensitivity), F1-score, and AUC-ROC.
The ROC curve is defined by:
\begin{equation}
\text{TPR} = \frac{TP}{TP + FN}, \quad
\text{FPR} = \frac{FP}{FP + TN}
\end{equation}
AUC provides a threshold-independent measure of discriminative performance.

% =========================
\section{Results}

On the held-out test set of 162 subjects, the proposed model achieves:

\begin{itemize}
    \item Accuracy: 0.679
    \item Precision: 0.505
    \item Recall (Sensitivity): 0.943
    \item F1-score: 0.658
    \item AUC-ROC: 0.824
\end{itemize}

Only 3 out of 53 dementia cases are misclassified as normal, corresponding to a false negative rate of 5.7\%.
Figure~\ref{fig:roc} shows the ROC curve, and Figure~\ref{fig:cm} presents the confusion matrix.

\begin{figure}[htbp]
\centering
\includegraphics[width=0.65\linewidth]{figures/roc_curve_plotly.png}
\caption{ROC curve on the held-out test set (AUC = 0.824).}
\label{fig:roc}
\end{figure}

\begin{figure}[htbp]
\centering
\includegraphics[width=0.55\linewidth]{figures/confusion_matrix_plotly.png}
\caption{Confusion matrix on the test set.}
\label{fig:cm}
\end{figure}

% =========================
\section{Discussion}

The results demonstrate that aggressive class weighting effectively prioritizes dementia detection, achieving high sensitivity at the cost of reduced specificity.
This trade-off is appropriate for screening applications, where missing a dementia case is more detrimental than flagging a healthy subject for further evaluation.

Compared to prior work~\cite{liu2018landmark, oh2019classification, korolev20173d}, the proposed model achieves competitive AUC while maintaining a simpler architecture and fewer parameters.
The moderate accuracy reflects the intentionally biased decision boundary favoring recall.

\section{Limitations and Future Work}

Limitations include the modest dataset size, binary classification setting, and lack of external validation.
Future work will explore attention mechanisms, multi-class classification including mild cognitive impairment, explainability via saliency maps, and evaluation on public datasets such as ADNI and OASIS.

% =========================
\section{Conclusion}

We presented a screening-oriented 3D CNN for Alzheimer's disease detection from structural MRI.
By prioritizing sensitivity through class-weighted training, the model achieves high recall (0.943) and competitive discriminative performance (AUC = 0.824).
These results demonstrate the potential utility of the approach as an automated first-pass screening tool to support clinical workflows.

\bibliographystyle{ieeetr}
\bibliography{references}

\end{document}